This procedural project is a part of the C\+O\+P2001 curriculum

\subsection*{Demonstration}

The first user enters credentials and is made an administrator. The username is set by the first letter of the first name concatenated with the last name in all lowercase. 

New products can be added to the product catalog, unit production can be tracked, new employee login information can be set if they have a unique username, production statistics can be displayed, and the user is able to find the production number by entering a serial number. 

\subsection*{Flowchart}



\subsubsection*{Sprint 1 Backlog (S\+BL)}

\subparagraph*{1. Output a greeting and menu in main.}

\subsubsection*{Sprint 2 Backlog (S\+BL) -\/ Conditions, Loops, Functions, Files intro.}

\subparagraph*{1. Create a working menu}


\begin{DoxyItemize}
\item The only menu options necessary for this sprint are Log Production and Exit.
\item Only accept values that are listed in the menu.
\item If the user selects a different value, they are asked to choose again.
\item Menus is printed from a show\+Menu function that is called from main.
\item Menu choices result in calls to functions.
\item After output of menu choice, show menu again until user chooses to exit. \subparagraph*{2. Add functionality to log production for the products from the previous sprint.}
\end{DoxyItemize}


\begin{DoxyItemize}
\item The user should be able to choose which of the products was produced.
\item Able to enter how many of the products were produced.
\item A record should be output that displays the production number, manufacturer, product name, item type and serial number.
\item Production\+Number should be unique, sequential for all products, and automatically assigned.
\item The Serial\+Number should start with the first three letters of the Manufacturer, two letter item\+Type code, then five digits (leading 0s if necessary that are unique and sequential for the item type. The entire Serial Number should be automatically assigned.
\end{DoxyItemize}

\subsubsection*{Sprint 3 -\/ Arrays and Vectors, Searching and Sorting}

\subparagraph*{1. Store the product line in a collection (parallel vectors).}


\begin{DoxyItemize}
\item Use code to populate some products.
\item Allow the user to add products.
\item All user to view the product line. \subparagraph*{2. When logging production, the user should be prompted to enter the product produced from the product line.}
\end{DoxyItemize}

\subparagraph*{3. Store the production log in a collection (parallel vectors).}

\subparagraph*{4. Output the product line sorted by name.}

\subparagraph*{5. Given a Serial Number, output the Production Number}

\subsubsection*{Sprint 4 -\/ Pointers, Characters, Strings, Recursion}

\subparagraph*{1. Create an Employee account.}


\begin{DoxyEnumerate}
\item Allow user to input full name in format First\+Name Last\+Name.
\end{DoxyEnumerate}
\begin{DoxyItemize}
\item Generate user id, which is their first initial and surname in all lowercase
\item Optional\+: Don\textquotesingle{}t allow duplicate user names.
\end{DoxyItemize}
\begin{DoxyEnumerate}
\item Allow user to input a password.
\end{DoxyEnumerate}
\begin{DoxyItemize}
\item The password must contain at least one digit, at least one lowercase letter, and at least one uppercase letter. The password cannot contain a space or any other symbols.
\item Use a recursive function to encrypt the password. \subparagraph*{2. Optional\+: Require users to log in. Track which employee recorded production.}
\end{DoxyItemize}

\subsubsection*{Sprint 5 -\/ Data Structures, Structs, Files}

\subparagraph*{1. Store product line, production records, and production statistics in structs.}

\subparagraph*{3. Save the production log to a text file name Production\+Log.\+csv}

\subparagraph*{4. Save the user names and encrypted passwords to a text file name Users.\+txt}

\subparagraph*{5. Load data from the files into vectors of structs when the program starts.}

\subsubsection*{Sprint 6 -\/ Quality Analysis}

\subparagraph*{1. Output production statistics such as total number of items produced and number of items of each type.}

\subparagraph*{2. Clear all warnings from Inspect Code.}

\subparagraph*{3. Format code using standard style guidelines.}

\subparagraph*{4. Enhance documentation\+:}


\begin{DoxyItemize}
\item Javadoc style comments for Doxygen
\item Comments in code
\item R\+E\+A\+D\+ME, create animated gif of program running \subparagraph*{5. Deployment}
\end{DoxyItemize}